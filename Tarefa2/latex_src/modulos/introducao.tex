\section{Introdução}\label{sec:introducao} %%%%%%%%%%%%%%%%%%%%%%
Uma definição sucinta para ``chatterbot'' é: um chatbot\footnote{A palavra \emph{chatbot} significa (do Inglês): \emph{chat}: bate-papo + \emph{bot}: abreviação de ``robô''} é um programa de computador capaz de realizar uma conversação auditiva ou textual \cite{blog_chatterbot} com o propósito de levar o interlocutor (uma pessoa) a pensar que está falando com outra pessoa, mesmo que brevemente.
Essa conversa pode ter vários propósito, tais como: encorajar clientes, facilitar acessos, automatizar processos, ensinar, assistir, entreter, engajar etc. A flexibilidade desse tipo de sistema é um dos fatores para a crescente demanda e desenvolvimento da área.

A interação é possível devido ao avanço da Inteligência Artificial que propõem formas de representar o conhecimento, estruturar bases de conhecimentos e processar a linguagem natural, conceitos essenciais para a elaboração de um chatterbot.
Em termos gerais, um chatterbot pode ter uma mecânica simples, reconhecendo estímulos bem definidos -- funcionando basicamente como um agente reativo simples\footnote{Agentes que respondem à percepções, interpreta a entrada, verifica a regra correspondente e age.\cite{wiki_agente_inteligente}} -- ou avançada, que reconhece, interpreta, avalia, aprende e responde de forma dinâmica (menos pré-definida). Além de poder ser incorporado em diversas plataformas, facilitando o seu acesso e ter uma interface agradável.